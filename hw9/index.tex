% Options for packages loaded elsewhere
\PassOptionsToPackage{unicode}{hyperref}
\PassOptionsToPackage{hyphens}{url}
%
\documentclass[
]{article}
\title{R Notebook}
\author{}
\date{\vspace{-2.5em}}

\usepackage{amsmath,amssymb}
\usepackage{lmodern}
\usepackage{iftex}
\ifPDFTeX
  \usepackage[T1]{fontenc}
  \usepackage[utf8]{inputenc}
  \usepackage{textcomp} % provide euro and other symbols
\else % if luatex or xetex
  \usepackage{unicode-math}
  \defaultfontfeatures{Scale=MatchLowercase}
  \defaultfontfeatures[\rmfamily]{Ligatures=TeX,Scale=1}
\fi
% Use upquote if available, for straight quotes in verbatim environments
\IfFileExists{upquote.sty}{\usepackage{upquote}}{}
\IfFileExists{microtype.sty}{% use microtype if available
  \usepackage[]{microtype}
  \UseMicrotypeSet[protrusion]{basicmath} % disable protrusion for tt fonts
}{}
\makeatletter
\@ifundefined{KOMAClassName}{% if non-KOMA class
  \IfFileExists{parskip.sty}{%
    \usepackage{parskip}
  }{% else
    \setlength{\parindent}{0pt}
    \setlength{\parskip}{6pt plus 2pt minus 1pt}}
}{% if KOMA class
  \KOMAoptions{parskip=half}}
\makeatother
\usepackage{xcolor}
\IfFileExists{xurl.sty}{\usepackage{xurl}}{} % add URL line breaks if available
\IfFileExists{bookmark.sty}{\usepackage{bookmark}}{\usepackage{hyperref}}
\hypersetup{
  pdftitle={R Notebook},
  hidelinks,
  pdfcreator={LaTeX via pandoc}}
\urlstyle{same} % disable monospaced font for URLs
\usepackage[margin=1in]{geometry}
\usepackage{graphicx}
\makeatletter
\def\maxwidth{\ifdim\Gin@nat@width>\linewidth\linewidth\else\Gin@nat@width\fi}
\def\maxheight{\ifdim\Gin@nat@height>\textheight\textheight\else\Gin@nat@height\fi}
\makeatother
% Scale images if necessary, so that they will not overflow the page
% margins by default, and it is still possible to overwrite the defaults
% using explicit options in \includegraphics[width, height, ...]{}
\setkeys{Gin}{width=\maxwidth,height=\maxheight,keepaspectratio}
% Set default figure placement to htbp
\makeatletter
\def\fps@figure{htbp}
\makeatother
\setlength{\emergencystretch}{3em} % prevent overfull lines
\providecommand{\tightlist}{%
  \setlength{\itemsep}{0pt}\setlength{\parskip}{0pt}}
\setcounter{secnumdepth}{-\maxdimen} % remove section numbering
\ifLuaTeX
  \usepackage{selnolig}  % disable illegal ligatures
\fi

\begin{document}
\maketitle

\hypertarget{modified-9.-each-of-150-newly-manufactured-items-is-examined-and-the-numberof-scratches-per-item-is-recorded.-the-items-are-supposed-to-be-free-of-scratches.-hereis-the-data}{%
\section{1. §6.1 Modified \#9. Each of 150 newly manufactured items is
examined, and the numberof scratches per item is recorded. (The items
are supposed to be free of scratches.) Hereis the
data:}\label{modified-9.-each-of-150-newly-manufactured-items-is-examined-and-the-numberof-scratches-per-item-is-recorded.-the-items-are-supposed-to-be-free-of-scratches.-hereis-the-data}}

\hypertarget{a.-find-an-unbiased-estimator-of-mu-and-compute-the-estimate-for-the-data.-hint-ex-mu-forxpoisson-so-ex-hint-2you-can-see-that-2-is-the-mostfrequent-observed-count-so-your-estimate-formu-should-probably-be-pretty-close-to-2.}{%
\subsection{\texorpdfstring{a. Find an unbiased estimator of \[\mu\] and
compute the estimate for the data. {[}Hint: E(X) =\[\mu\] forXPoisson,
so E(X) = ?{]} {[}Hint 2:You can see that ``2'' is the mostfrequent
observed count, so your estimate for\[\mu\] should probably be pretty
close to
2.{]}}{a. Find an unbiased estimator of \textbackslash mu and compute the estimate for the data. {[}Hint: E(X) =\textbackslash mu forXPoisson, so E(X) = ?{]} {[}Hint 2:You can see that ``2'' is the mostfrequent observed count, so your estimate for\textbackslash mu should probably be pretty close to 2.{]}}}\label{a.-find-an-unbiased-estimator-of-mu-and-compute-the-estimate-for-the-data.-hint-ex-mu-forxpoisson-so-ex-hint-2you-can-see-that-2-is-the-mostfrequent-observed-count-so-your-estimate-formu-should-probably-be-pretty-close-to-2.}}

\[
\lambda = \frac{\sum i*X_i}{n} = \frac{14*0+35*1+43*2+28*3+4*17+5*10+6*2+1*7}{150} = \frac{342}{150} = 2.28
\] \[
E(\bar{X})=E(X) = \lambda=2.28
\] \#\# b. What is the standard deviation (standard error) of youur
estimator? Compute the estimated standard error. {[}Hint std\^{}2 = mean
for X{]} For poisson distribution \[V(x) = \lambda = 2.28\],
\[V(\bar{x}) = \frac{V(x)}{n} = \frac{2.28}{150} = 0.0152\]

\hypertarget{of-n1-randomly-selected-male-smokers-x1-smoked-filter-cigarettes-whereasof-n2-randomly-selected-female-smokers-x2-smoked-filter-cigarettes.-let-p1-and-p2-denote-the-probabilities-that-a-randomly-selected-male-and-femal-respectively-smokefilter-cigarettes.}{%
\section{2. 6.1 \#11. Of n1 randomly selected male smokers, X1 smoked
filter cigarettes, whereasof n2 randomly selected female smokers, X2
smoked filter cigarettes. Let p1 and p2 denote the probabilities that a
randomly selected male and femal, respectively, smokefilter
cigarettes.}\label{of-n1-randomly-selected-male-smokers-x1-smoked-filter-cigarettes-whereasof-n2-randomly-selected-female-smokers-x2-smoked-filter-cigarettes.-let-p1-and-p2-denote-the-probabilities-that-a-randomly-selected-male-and-femal-respectively-smokefilter-cigarettes.}}

\hypertarget{a.-show-that-x1n1x2n2-is-an-unbiased-estimator-for-p1-p2.}{%
\subsection{a. Show that (X1/n1−X2/n2) is an unbiased estimator for p1 −
p2.}\label{a.-show-that-x1n1x2n2-is-an-unbiased-estimator-for-p1-p2.}}

\[
p_1 - p_2 = E(x_1)-E(x_2) = \frac{x_1}{n_1}-\frac{x_2}{n_2}
\] \#\# b. What is the standard error of the estimator in (a)?

\[
Var(p_1-p_2) = Var(p_1)+Var(P_2) = n_1p_1(1-p_1) +n_2p_2(1-p_2)
\] \[
Standard Error =\sqrt{Var(p_1-p_2)} = \sqrt{n_1p_1(1-p_1) +n_2p_2(1-p_2)}
\]

\hypertarget{c.-how-would-you-use-the-observed-values-x1-and-x2-to-estimate-the-standard-error-of-your-estimator}{%
\subsection{c.~How would you use the observed values x1 and x2 to
estimate the standard error of your
estimator?}\label{c.-how-would-you-use-the-observed-values-x1-and-x2-to-estimate-the-standard-error-of-your-estimator}}

I would find p1 and p2 using \[p_i = \frac{x_i}{n_i}\]. And then plug
into formula found in 2b.

\hypertarget{d.-if-n1-n2-200x1-127-and-x2-176-use-the-estimator-of-a-to-obtain-a-nestimate-of-p1p2.}{%
\subsection{d.~If n1 = n2 = 200,X1 = 127, and x2= 176, use the estimator
of (a) to obtain a nestimate of
p1−p2.}\label{d.-if-n1-n2-200x1-127-and-x2-176-use-the-estimator-of-a-to-obtain-a-nestimate-of-p1p2.}}

\[
p1−p2 = \frac{127}{200} -\frac{176}{200} = -0.245
\]

\hypertarget{e.-use-the-result-of-part-c-and-the-data-of-part-d-to-estimate-the-standard-error-ofthe-estimator.}{%
\subsection{e. Use the result of part (c) and the data of part (d) to
estimate the standard error ofthe
estimator.}\label{e.-use-the-result-of-part-c-and-the-data-of-part-d-to-estimate-the-standard-error-ofthe-estimator.}}

\[
p_1 = \frac{127}{200}=0.635
\] \[
p_2 = \frac{176}{200} = 0.88
\] \[
SE = \sqrt{200*0.635(1-0.635) + 200*0.88(1-0.88)} = 8.21
\]

\hypertarget{for-each-of-the-following-estimators-of-mean-if}{%
\section{3. For each of the following estimators of mean if
\ldots{}}\label{for-each-of-the-following-estimators-of-mean-if}}

\hypertarget{a.}{%
\subsection{a.}\label{a.}}

\[\bar{X_5}\] is unbiased for \mu{]} because normal distribution is
symetric about the mean. \#\# b. \[X_1-X_2+X_3-X_4+X_5\] is unbiased as
\[X_n=X_{n+1}\],therefore everything except for one X cancels out. \#\#
c. \[
\frac{X_1+2X_2+3X_3+4X_4+5X_5}{15} = \frac{15X}{15} = X
\] Therefore \[\frac{X_1+2X_2+3X_3+4X_4+5X_5}{15}\] is an unbiased
estimate for \[\hat{\mu}\]

\hypertarget{d.}{%
\subsection{d.~}\label{d.}}

\[1.01\bar{X}_5\] is not an unbiased estimate for \[\hat{\mu}\].
Standard error is \[
0.01*\bar{X}_5
\] \# 4. Let Xi =, i=1,2..n be continuous random variables with pdf f(x)
= \{Theta\}exp(-Theta x), x\textgreater0, We're given that
Theta\textgreater0. The Xi's are independent.

\hypertarget{a.-find-the-method-of-moments-estimator-of-theta-theta_mom}{%
\subsection{a. Find the Method of Moments estimator of Theta,
Theta\_mom}\label{a.-find-the-method-of-moments-estimator-of-theta-theta_mom}}

\[
E(X) = \int_0^\infty xf(x)dx = \int_0^\infty x\theta e^{-\theta x}dx = \frac{1}{\theta} = \bar{x}
\] \[
\theta_{Mom} = \frac{1}{\bar{x}}
\] \#\# b. Is Theta\_mom unbiased for Theta

\[\theta_{Mom}\] is unbiased assuming that n is sufficiently large, by
the central limit theorem as distribution of \[\theta_{Mom}\] the normal
distribution.

\hypertarget{c.-find-the-likelihood-functionlux3b8.}{%
\subsection{c.~Find the likelihood
function,L(θ).}\label{c.-find-the-likelihood-functionlux3b8.}}

\hypertarget{d.-find-the-maximum-likelihood-estimator-of-theta-theta_mle}{%
\subsection{d.~Find the Maximum Likelihood estimator of Theta,
Theta\_mle}\label{d.-find-the-maximum-likelihood-estimator-of-theta-theta_mle}}

\hypertarget{e.-is-theta_mle-unbiased-for-theta}{%
\subsection{e. Is Theta\_mle unbiased for
Theta}\label{e.-is-theta_mle-unbiased-for-theta}}

\hypertarget{let-yibernullip-i12..n}{%
\section{5. Let Yi=Bernulli(p) i=1,2,..n}\label{let-yibernullip-i12..n}}

\hypertarget{a.-find-the-method-of-moments-estimator-ofp.}{%
\subsection{a. Find the Method of Moments estimator
ofp.}\label{a.-find-the-method-of-moments-estimator-ofp.}}

\hypertarget{b.-find-the-maximum-likelihood-estimator-ofp}{%
\subsection{b. Find the Maximum Likelihood Estimator
ofp}\label{b.-find-the-maximum-likelihood-estimator-ofp}}

\hypertarget{c.-are-either-of-these-estimators-unbiased-forp}{%
\subsection{c.~Are either of these estimators unbiased
forp?}\label{c.-are-either-of-these-estimators-unbiased-forp}}

\hypertarget{chapter-6-supplementary-exercise-32}{%
\section{6. Chapter 6 Supplementary exercise \#
32}\label{chapter-6-supplementary-exercise-32}}

\hypertarget{a.-letx1xnbe-a-random-sample-from-a-uniform-distribution-on-0ux3b8.-hereux3b8-0.-then-the-mle-ofux3b8isux2c6ux3b8y-maxx1xn.-it-takes-some-work-toshow-this-is-the-mle-the-text-discusses-it-in-example-6.22.-you-are-not-asked-toshow-this-take-it-as-a-given.-show-that-the-pdf-ofy-is}{%
\subsection{a. LetX1,\ldots,Xnbe a random sample from a uniform
distribution on {[}0,θ{]}. (Here,θ \textgreater0.) Then the MLE
ofθisˆθ=Y= max\{X1,\ldots,Xn\}. (It takes some work toshow this is the
MLE; the text discusses it in Example 6.22. You are not asked toshow
this; take it as a given.) Show that the pdf ofY
is}\label{a.-letx1xnbe-a-random-sample-from-a-uniform-distribution-on-0ux3b8.-hereux3b8-0.-then-the-mle-ofux3b8isux2c6ux3b8y-maxx1xn.-it-takes-some-work-toshow-this-is-the-mle-the-text-discusses-it-in-example-6.22.-you-are-not-asked-toshow-this-take-it-as-a-given.-show-that-the-pdf-ofy-is}}

\[
f_Y(y) = \frac{ny^{n-1}}{\Theta^n}
\] \#\# b. Use the result of part (a) to show that the MLE is biased,
but y(n+1)/n is unbiased for Theta.

\hypertarget{the-birthday-problem.-mars-experiences-just-under-667-martian-days-and-nights-eachtime-it-revolves-around-the-sun.-lets-pretend-its-exactly-667-days.-little-martianfur-babies-have-birthdays-that-are-uniformly-distributed-throughout-the-martian-year.letenbe-the-event-that-in-a-group-ofnrandomly-selected-martian-fur-babies-no-twoshare-the-same-martian-birthday-i.e.-all-of-them-have-distinct-birthdays}{%
\section{7. The birthday problem. Mars experiences just under 667
Martian days and nights eachtime it revolves around the sun. Let's
pretend it's exactly 667 days. Little Martianfur-babies have birthdays
that are uniformly distributed throughout the Martian year.LetEnbe the
event that in a group ofnrandomly selected Martian fur-babies, no
twoshare the same Martian birthday; i.e., all of them have distinct
birthdays}\label{the-birthday-problem.-mars-experiences-just-under-667-martian-days-and-nights-eachtime-it-revolves-around-the-sun.-lets-pretend-its-exactly-667-days.-little-martianfur-babies-have-birthdays-that-are-uniformly-distributed-throughout-the-martian-year.letenbe-the-event-that-in-a-group-ofnrandomly-selected-martian-fur-babies-no-twoshare-the-same-martian-birthday-i.e.-all-of-them-have-distinct-birthdays}}

\hypertarget{a.-find-pe2-the-probability-that-in-a-group-of-2-randomly-selected-martian-fur-babies-the-fur-babies-have-2-distinct-birthdays.}{%
\subsection{a. Find P(E2), the probability that in a group of 2 randomly
selected Martian fur-babies, the fur-babies have 2 distinct
birthdays.}\label{a.-find-pe2-the-probability-that-in-a-group-of-2-randomly-selected-martian-fur-babies-the-fur-babies-have-2-distinct-birthdays.}}

\[
1/667
\]

\hypertarget{b.-findpe5.-findpec5-the-probability-that-in-a-group-of-5-randomly-selectedfur-babies-at-least-two-will-share-the-same-birthday}{%
\subsection{b. FindP(E5). FindP(EC5), the probability that in a group of
5 randomly selectedfur-babies, at least two will share the same
birthday}\label{b.-findpe5.-findpec5-the-probability-that-in-a-group-of-5-randomly-selectedfur-babies-at-least-two-will-share-the-same-birthday}}

\[
1-\frac{667}{667}*\frac{666}{667}*\frac{665}{667}*\frac{664}{667}*\frac{663}{667}=1-0.985 = 0.015
\]

\hypertarget{c.-find-the-formula-forpen.}{%
\subsection{c.~Find the formula
forP(En).}\label{c.-find-the-formula-forpen.}}

\[
P(E_n) = 1-\frac{\frac{667!}{(667-n)!}}{(667)^n} = 1-\frac{667!}{(667-n)!*667^n}
\]

\hypertarget{extra-credit-referring-to-problem-7-for-which-group-sizen-is-pecn12-in-otherwords-how-many-fur-babies-does-it-take-before-theres-a-better-than-50-50-chance-that-two-or-more-of-them-will-share-the-same-birthday}{%
\section{8. (Extra credit) Referring to problem 7: For which group sizen
is P(ECn)\textgreater1/2? In otherwords, how many fur-babies does it
take before there's a better than 50-50 chance that two or more of them
will share the same
birthday?}\label{extra-credit-referring-to-problem-7-for-which-group-sizen-is-pecn12-in-otherwords-how-many-fur-babies-does-it-take-before-theres-a-better-than-50-50-chance-that-two-or-more-of-them-will-share-the-same-birthday}}

It takes 31 babies for there to be a 50-50 chance that they will share
the same birthday

\end{document}
